\newpage

\title{Описание использованного метода}{\begin{center}
    Описание метода
\end{center}} \\

Метод Милна относится к многошаговым методам и представляет один из методов прогноза и коррекции. \\
Решение в следующей точке находится в два этапа. На первом этапе осуществляется по специальной формуле прогноз значения функции, а затем на втором этапе - коррекция полученного значения. Если полученное значение $y $ после коррекции существенно отличается от спрогнозированного, то проводят еще один этап коррекции. Если опять имеет место существенное отличие от предыдущего значения (т.е. от предыдущей коррекции), то проводят еще одну коррекцию и т.д. Однако очень часто ограничиваются одним этапом коррекции. \\
Пусть для уравнения $y' = f(x,y)$ кроме начального условия $y(x_0) = y_0$ известен "начальный отрезок", то есть значения искомой функции $y(x_i) = y_i$ в точках $x_i = x_0 + i\cdot h, (i = 1,2,3)$, данные значения можно найти каким-либо одношаговым методом (в дальнейшем используется Метод Рунге-Кутты 4-го порядка).
\newline
\newline
Метод Милна имеет следующие вычислительные формулы: 
\newline
а) этап предположения (прогноза):
$$(y_{i+1})^{prog} = y_{i-3}+4\cdot\frac{h}{3}\cdot(2f_{i-2}-f_{i-1}+2f_i),$$
где для компактности записи использовано следующее обозначение $f_i = f(x_i, y_i)$;
\newline
б) этап коррекции:
$$(y_{i+1})^{corr} = y_{i-1}+\frac{h}{3}\cdot(f_{i+1}+4f_{i}+(y_{i+1})^{prog}),$$
Абсолютная погрешность определяется по формуле
$$\epsilon \approx \frac{(y_{i+1})^{corr}-(y_{i+1})^{prog}}{29}$$
