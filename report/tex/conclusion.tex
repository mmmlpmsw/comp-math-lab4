\newpage
Вывод: 
\textbf{метод Милна} относится к многошаговым методам и представляет один из методов прогноза и коррекции, как и \textbf{метод Адамса}. Разница между методами Адамса и Милна заключается в использовании разных формул прогноза и коррекции: в методе Милна в качестве интерполяционного полинома используется полином Ньютона, в методе Адамса - полином Лагранжа. Оба этих метода имеют четвертый порядок точности. Поскольку методы являются многошаговыми, для вычисления значения нам необходимо знать результаты нескольких предыдущих шагов, поэтому невозможно, если так можно выразиться, запустить метод: для этого необходимо предварительно получить одношаговыми методами первые три точки. Кроме того, методы прогноза и коррекции требуют  дополнительного расхода памяти - поскольку для них требуются данные о предыдущих точках. \\ 
Далее перейдем к сравнению одношаговых методов. \\
\textbf{Метод Эйлера} основан на получении каждого следующего значения $y$ из предыдущего: $$y_{i+1}=y_i+h\cdot f(x_i;y_i). $$ 
Данный метод имеет большую погрешность, которая, к тому же, накапливается на каждом шаге. Порядок точности данного метода - первый. \\
\textbf{Усовершенствованный метод Эйлера} отличается от обычного тем, что значение правой части уравнения берется равным среднему арифметическому между $f(x_i;y_i)$ и $f(x_{i+1};y_{i+1})$, то есть
$$y_{i+1}=y_i+\frac{h}{2} \cdot (f(x_i;y_i)+f(x_{i+1};y_{i+1})),$$
затем вычисляется первое приближение $\widetilde{y}_{i+1} = y_i+h \cdot f(x_i,y_i)$, затем подставляем значение в формулу выше и находим уточненное значение 
$$y_{i+1}=y_i+\frac{h}{2} \cdot (f(x_i;y_i)+f(x_{i+1};\widetilde{y}_{i+1})).$$ 
Данный метод точнее метода Эйлера и имеет второй порядок точности. \\
\textbf{Метод Рунге-Кутта} имеет несколько разновидностей, различающихся порядком точности. В этих методах допускается вычисление правых частей не только в точках сетки, но и в некоторых промежуточных точках. \\
Рассмотрим \textbf{метод Рунге-Кутта 4-го порядка точности}. В данном методе вводятся 4 вспомогательные величины $(k_0, k_1, k_2, k_3)$ и вычисление координат очередной точки сетки происходит исходя из известных координат предыдущей точки: $$y_{i+1}=y_i+ \frac{1}{6} \cdot (k_0+2\cdot k_1+2 \cdot k_2+k_3) , i = 0, 1, ... $$ (формулы вычисления вспомогательных величин $k_0, k_1, k_2, k_3$ опущены). Таким образом, данный метод требует на каждом шаге четырехкратного вычисления правой части уравнения. Метод Рунге-Кутта требует большого объема вычислений, но имеет повышенную точность, что позволяет проводить вычисления с большим шагом.\\
\newline
При одинаковом шаге метод Эйлера и усовершенствованный метод Эйлера менее точные, в отличие от метода Рунге-Кутта четвёртого порядка. \\